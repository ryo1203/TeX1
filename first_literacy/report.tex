\documentclass{jsarticle}
\renewcommand{\baselinestretch}{1.2}
\begin{document}
\pagestyle{empty}
\title{一類リテラシレポート課題}
\author{石塚 伶\\18B01162}
\date{2018年4月20日 提出}
\maketitle
\large{自分は数学系に行って将来は数学に関する仕事に就きたいがその様なもののうちどの仕事に就くとしてもとりあえず数学をしっかりと修めなければならない。数学は講義形態で教えてもらうことが大学では今のところ多いが、やはり自分で勉強していかなければ理解が深まらないだろうと考えている。つまり国際化の流れから必要な技術を考える以前の技術としてまずは自分で数学書を読みそしてそれを理解することが必要なものだろう。

まず、数学書を読みそれを理解したと言えるのはその概念を他人にしっかりと教えられたらだと思っている。そこで、理解した後日本語で伝えることはしっかりと理解してさえいれば簡単なことかもしれないがそれを英語で日本人以外に伝えるとしたらどうだろうか。そのためには日本語で理解したものを英語に訳せるほどその分野に精通していなければならずそしてその英語を形にするための英語力が必要だと考えられる。つまり、理解したものを英語として出力することによって自分の理解も深まると思う。そのつながりで次は日本語に訳される前の最新の洋書を読めたりするのではないだろうか。

また、もし数学者など学会での発表をしたりされたりする立場になったときそこでの言語は英語であると聞くのでそこでもやはり英語力というのは必要な技能の一つなのではないだろうか。自分の思う限りそこでの英語は流暢であることよりも自分の伝えたいことをしっかりと伝え数学として論理の流れが明確かつ正確であることが必要なのだと思う。そのような英語は留学やネイティブとの会話で身につくものではなく上で示したように自分の数学の理解力とそれをいかに表現するかの能力によって作られるものなのではないだろうか。

結局のところ将来のために必要な国際化の流れを汲む能力は深まった自分の理解を正確に表現できる英語力なのではないかと思う。}
\end{document}
