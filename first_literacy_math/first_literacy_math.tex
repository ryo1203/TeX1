\documentclass[dvipdfmx]{jsarticle}
\usepackage{mymacros}
\usepackage{amsmath,amssymb,amsthm}
\usepackage{tcolorbox}
\usepackage{mathrsfs}
\usepackage{tikz}
\usepackage{multicol}
\newtheorem{theo}{定理}
\newtheorem{defi}{定義}
\newtheorem{lemm}{補題}
\newtheorem{plob}{演習問題}
\newtheorem{axio}{公理}
\title{1類リテラシ・1類特別講義レポート課題(数学系)}
\author{18B01162 石塚伶}
\date{}
\begin{document}
\maketitle
\subsection{概要}
\titlebox{プラトーの問題}{与えられた閉曲線を境界に持つ曲面積が最小の局面は存在するか。}
この問題に対して、存在することはシャボン玉の膜の実験により言えそうだがそれを数学でどのように証明すれば良いのかを考えた。
\begin{multicols}{2}
    まずは三次元から一つ次元を落として、与えられた二点を結ぶ曲線のうち長さが最小の曲線が存在するかを考える。

    これはその二点を結ぶ直線であることが容易にわかるがまずそれを微分方程式の形で証明する。

    二点 $A(a,\alpha) B(b,\beta)$をとったときにその二点を結ぶ最小の曲線 $f(x)$とまた他の曲線 $g(x)$を取ってくる。(このとき $g(x)$の長さは $f(x)$と同じかそれ以上となる)
    よって $f$の最小性から
    \[
    \int_a^b \sqrt{1 + g'(x)^2} \, dx \, \geq \, \int_a^b \sqrt{1 + f'(x)^2} \, dx
    \]
    がどんな $g$に対しても成り立つ。

    またここで $h(x)$: $a,b$の近くで0である任意の $[a,b]$上の関数を取ってきて
    \[
    y = f(x) + th(x) \, (t \in \mathbb{R})
    \]
    を考えると

    $t = 0 \Rightarrow y = f(x)$で長さが最小、
    $t$を動かすと $h(x)$の分だけ $f(x)$が変化する。

    そこでこの曲線の長さは
    \[
    \int_a^b \sqrt{1 + (f'(x) + th'(x))^2} \, dx
    \]
    となりこれは $t$の関数となっている。
    上述したとおりこの関数は $t = 0$で最小になっているからその点での導関数の値は0になっている。つまり、この場合微分が積分の中に入れられることから
    \begin{eqnarray*}
      \frac{d}{dt}\int_a^b \sqrt{1 + (f'(x) + th'(x))^2} \, dx & = & 0 \\
      \int_a^b \dell{}{t} \sqrt{1 + (f'(x) + th'(x))^2} \, dx & = & 0 \\
      \left. \int_a^b \frac{1}{2} \frac{2(f'(x) + th'(x))h'(x)}{\sqrt{1 + (f'(x) + th'(x))^2}} \, dx \right|_{t = 0} & = & 0 \\
      \int_a^b \frac{f'(x) h'(x)}{\sqrt{1 + f'(x)^2}} & = & 0
    \end{eqnarray*}
    部分積分と、 $h(x)$の定義から $h(a) = h(b) = 0$より
    \begin{eqnarray*}
      \left[ \frac{f'(x)}{\sqrt{1 + f'(x)^2}} h(x) \right]_a^b - \int_a^b \left( \frac{f'(x)}{\sqrt{1 + f'(x)^2}} \right)' h(x) \, dx & = & 0 \\
      \int_a^b \left( \frac{f'(x)}{\sqrt{1 + f'(x)^2}} \right)' h(x) \, dx & = & 0 \\
    \end{eqnarray*}
    そして、 $h(x)$は $x = a,b$で $0$になるという条件以外は任意なので上の式が成り立つためには
    \[
    \left( \frac{f'(x)}{\sqrt{1 + f'(x)^2}} \right)' = 0
    \]
    が必要。この左辺は曲率と呼ぶ。
    この条件より $f''(x) = 0$が必要となりこれは $A,B$を結ぶ直線にほかならない。よって長さが最小となる曲線は与えられた二点を結ぶ直線であることがわかった。
    \\
    \\
    三次元でも同様に曲面積が最小となる $z = f(x,y)$となる関数と $xy$平面上の領域 $U$を考えれば積分の式が
    \[
    \iint_U \sqrt{1 + f_x^2 + f_y^2} \, dxdy
    \]
    となって曲率が0になることが必要であるので結果として
    \[
    \left( \frac{f_x}{\sqrt{1 + f_x^2 + f_y^2}} \right)_x + \left( \frac{f_y}{\sqrt{1 + f_x^2 + f_y^2}} \right)_y = 0
    \]
    を満たす $f(x,y)$が曲面積を最小にする。

    この答えは曲面積を最小にする曲面が存在しているという仮定のもとでその曲面が満たす条件を述べただけなので存在するか否かに対しては答えていない。

    このような微分方程式の形で考えるものの他に微分形式といったものを使う場合もある。

    微分形式 $D$とは
    \[
    D = \{ f(x,y)dx + g(x,y)dy
    \]
    \[
     : f(x,y) , g(x,y)は二変数関数 \}
    \]
    のことをいう。
    $D^*$を $D$上の(線形)関数全体としたとき$(曲線の集合) \subset D^*$と定義する。

     曲線 $C$を $D$上に定義された関数と見て
    \[
    C(fdx + gdy) = \int_C f \, dx + \int_C g \, dy
    \]
    とする。 $x = x(s) , y = y(s)$としてパラメーター表示を行うと結果的に
    \[
    Cの長さ = \max_{|f|^2 + |g|^2 \leq 1} \left|C(fdx + gdy) \right| = |C|
    \]
    となることが分かる。よって、

    最小の長さの曲線を見つける。 $\Leftrightarrow D^* (\supset (曲線全体))$のなかで $|C|$を最小化するものを見つける。

    こととなりこれを三次元でも同様に行うことができる。
    \\
    \\
    これらはブラックホールの見かけの事象の地平線が極小平面でありこの面積がブラックホールの質量になっていることや、ナビエ・ストークス方程式の滑らかさの問題や、ポアンカレ予想に用いられた。
    \subsection{興味を覚えた点}
    曲線や曲面の存在を言うときに具体的な関数を出さずに間接的に存在をいっていることが数学らしく良いと思った。
    また、微分形式といった方法で人工的な演算を考えたとしてもそれによる結果が自然界の法則の解明に役立つことが不思議だった。
    曲面積や曲線の長さを最小にするときに曲率がでてくるのが突飛な気がしたがこれの次元を上げることで $n$次元の曲率を考えてそこでの最小性を議論できるのではないかと考えた。
    \subsection{この分野の将来予測}
    将来的には物理や宇宙に関わる場所でブラックホールの問題と同様に曲面積がでてくる問題に対して有効に働くのではないかと思った。
    \subsection{一類リテラシの感想(数学系)}
    1次元落として考えたものの実際の計算を行ってよくある数式がなくて理解できるものとしてではなく数式を用いて式変形を行いある程度の厳密性があったためわかりやすく興味が湧いた。
    最後の微分形式のあたりからは足早に進んでしまっていたがそれでも今までになかった数学へのアプローチの方法を見れてよかったしそれの応用もわかった。
    \clearpage
\end{multicols}
\end{document}
