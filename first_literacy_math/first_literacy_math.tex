\documentclass[dvipdfmx]{jsarticle}
\usepackage{mymacros}
\usepackage{amsmath,amssymb,amsthm}
\usepackage{tcolorbox}
\usepackage{tikz}
\begin{document}
\タイトル{一類リテラシ数学系講義}{}
\begin{tcolorbox}[colback=white,title=\textbf{プラトーの問題}]
  与えられた閉曲線を境界に持つ曲面積が最小の曲面は存在するか。
\end{tcolorbox}

\begin{table}[h]
  \begin{tabular}{rl}
    この問題の探求 $\longrightarrow$ & $\circ$ ブラックホールの質量定理 \\
     & $\circ$ 水などの流体方程式(ナビエ・ストークス方程式)の滑らかさについての問題 \\
     & $\circ$ ポアンカレ予想の解決 \\
  \end{tabular}
\end{table}

\section{おもちゃ問題}
\begin{tcolorbox}[colback=white,title=\textbf{おもちゃ問題}]
与えられた2点を境界に持つ長さが最小の曲線は存在するか。
\end{tcolorbox}
\[
\forall \epsilon > 0 : \exists \delta > 0
\]
\titlebox{タイトル}{本文}

\end{document}
