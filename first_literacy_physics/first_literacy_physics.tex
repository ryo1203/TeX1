\documentclass[dvipdfmx]{jsarticle}
\usepackage{mymacros}
\usepackage{amsmath,amssymb,amsthm}
\usepackage{tcolorbox}
\usepackage{mathrsfs}
\usepackage{tikz}
\usepackage{multicol}
\newtheorem{theo}{定理}
\newtheorem{defi}{定義}
\newtheorem{lemm}{補題}
\newtheorem{plob}{演習問題}
\newtheorem{axio}{公理}
\title{1類リテラシ・1類特別講義レポート課題(物理学系)}
\author{18B01162 石塚伶}
\date{}
\begin{document}
\maketitle
\subsection{概要}
前半は「重力波の観測」として重力波を観測するまでの流れや観測施設の説明、大人数で行うプロジェクトの意義を聞き、後半は「量子力学を使ったシュミレーションとセンサーの話」として量子シュミレーションとはどういうものか、そしてそれはどういったことに使われようとしているのかについて聞いた。

\subsubsection{前半 \textbf{重力波の観測}}

そもそも重力というものはニュートンが発見したが当時は物体と物体の間に働く力と言われていた。それがアインシュタインの一般相対性理論によって重力とは空間の歪みであって地球のようなものがあると重力場が歪むといった理論を展開した。重力波はブラックホールが星を吸い込むときの固有振動のようなものが生成したり、連星の合体のときに発生する。

重力波を見ることで次のようなものが分かる。
\begin{itemize}
  \item 超新星爆発
  \item ブラックホール
  \item パルサー
  \item 未知の天体
  \item 背景重力波(初期宇宙の量子的なゆらぎ)
  \item 中性子星・連星
\end{itemize}
そもそも重力波は連星を観測したときに電磁波よりも大きなエネルギーを消失していることが判明したことにより存在が予想された。
日本でも観測機器が作られたが結果的に米国のLIGOが先に見つけることとなる。
このとき見つけた重力波は太陽の30倍ほどの質量を持つブラックホールの衝突によって出来たものであって、同様のものが5例見つかった。
京大が発見した理論によると太陽の30倍の質量を持つブラックホールが一番多く存在しているということだったがそれとも一致する結果となった。

日本でもKAGRAといった重力波望遠鏡を作ったいる。これは騒音を減らすために地下にあり、ガラスではなくサファイアを使い冷やすことによって熱伝導効率を高めている。

またこのような大規模プロフェクトは個人で行うものとは異なる点が多い。
大型プロフェクトはやっていることを秘密にしなくてもいいのに対して小規模プロフェクトは予算規模も少ない分競争が激しくやっていることを秘密にすることが多い。

大規模プロジェクトであっても日米欧で雰囲気が異なり、アメリカはそのプロフェクトに携わっている全員が同じ方向を向いているため個人で好きなことがあまり出来ないがプロフェクトは真っ直ぐ進むようになる。
それに対して日本やヨーロッパは個人である程度自由になっているため自分だけで発見をしたり論文を書けたりするがその分プロフェクトはまっすぐ進まなくなっている。

\subsubsection{後半 \textbf{量子力学を使ったシュミレーションとセンサーの話}}

そもそも量子リュミレーションとは真空中を飛び交う原子気体にレーザーを照射するとその温度をμK台にまでさげることができ、冷却された原子気体は制御性の高い量子系として振る舞うためその冷却原子をつかって行うもの。

超電導体を作っているが今までは高温超伝導体であって、常温での超電導体は作られていない。また、その相互作用もある程度わかっているが計算不可能なレベルなのでこれと同じモデルを持ってきて量子系で計算させようとしている。

これにより自己位置決定の精度をcm単位で行うことができるかもしれないが電磁波を使うため妨害なども考えられる。そのため電磁波ではなくてジャイロスコープと加速度計を時間で2回積分することで位置決定を行おうとしている。
しかしジャイロスコープの精度が悪すぎるためその誤差を直している。

このような色々なことに量子は使え、テクノロジーを質的に変化させる原動力とな。

\subsection{興味をもった点}
重力波といった宇宙規模での現象と超伝導体など物質の規模での現象をともに物理の範囲で行っていることが面白かった。前半で言った大規模プロジェクトと個人で行う者の違いが実際に見てその違いが他のところではどのようになっているのか気になった。

また、重力波は更にこれから機器や研究手法の改善によってもっと多くのデータが集まると思い、そのデータによって今までの天文学とは異なる重力波を観測に使う重力波天文学がどのように発展していくのか気になった。

後半の量子力学を使うというところでは自分が漠然と思っていた量子力学が数式上の学問ではないかといったイメージとは違い観測機器などに実際に用いられるというところが興味深かった。

\subsection{この分野の将来予測}
上に書いたように電磁波の観測に変わり重力波を観測することによる天文学が発展していくと思う。
また、目視ができないブラックホールの観測を重力波で行うことによりいままで詳しくわかっていなかったブラックホール内部などもわかるのではないかと考えた。
また、ジャイロスコープのところで精度が上がることで新たな自己位置決定方法が確立されれば自動運転なども精度が高くなり広く広まると思う。
量子力学の応用が更に進むことでミクロの世界の変化をマクロの世界にまで広げて新たな機器が作られていくと思う。

\subsection{1類リテラシの感想(物理学系)}
前半と後半に分かれて分野も規模も違う人の話を聞けてよかった。
重力波と言った自分が少し知っている、最近になってでてきた学問と自分のまったく知らなかった量子シュミレーションの話が聞けたことで物理学としての幅広さを知れた。
どちらの先生とも自分の研究のこれからや野望を述べていて教える人ではなく学者としての一面が見えた。


\end{document}

\end{document}
