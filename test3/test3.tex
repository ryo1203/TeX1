\documentclass[dvipdfmx]{jsarticle}
\usepackage{mymacros}
\usepackage{amsmath,amssymb}
\usepackage{tcolorbox}
\usepackage{tikz}
\begin{document}
\タイトル{TikZの練習}{石塚 伶}
\begin{center}
  \begin{tikzpicture}[auto]
    \node (a) at (0,0) {$a$}; \node (b) at (1.2,0) {$b$};
    \draw [-to] (a) to node {$\scriptstyle f $} (b);
  \end{tikzpicture}
\end{center}
\clearpage
\タイトル{数式の練習}{石塚 伶}
\rotatebox{160}{\reflectbox{アインシュタインは $E=mc^2$ と言った。}}

\begin{tcolorbox}[colback=white,title=粋な枠]
  \[ \sqrt{\left( \frac{\displaystyle\int_0^\infty f(x)\,dx}{\displaystyle\int_0^\infty g(x)\,dx} \right)^n} \]
\end{tcolorbox}

文章中の和分$\sum_{k=0}^\infty$\\
文章外の和分
\begin{equation}
\textstyle\sum\limits_{k=0}^\infty a_k = a_1 + a_2 + \cdots \label{eq:inftysum}
\end{equation}
文章中の和分$\displaystyle\sum_{k=0}^\infty$
文章中の和分\[\sum\nolimits_{k=0}^\infty\]

\[\int\!\!\!\!\int\int\]

無限級数の和 (\ref{eq:inftysum}) \pageref{eq:inftysum}
\[\bigl| |x| + |y| \bigr|\]
$\left(x-f(x)\right) / \left(x+f(x)\right)$
$\bigl(x-f(x)\bigr) \bigm/ \bigl(x+f(x)\bigr)$

$\epsilon - \delta$論法

 ある関数$f(x)$が$x=a$で連続であるとは
\[\forall \epsilon >\!0 \quad \exists \delta >\!0 \quad s.t \quad \forall x \in \mathbb{R} \quad \bigl[\ |x - a|< \delta \Longrightarrow |f(x) - f(a)|< \epsilon\ \bigr]\]
が成り立つことをいう。

ある数列$a_n$が$\alpha$に収束するとは
\[\forall \epsilon > 0 \quad \exists N \in \mathbb{N} \quad s.t \quad \forall n \in \mathbb{N} \quad \bigl[n > N \Longrightarrow |a_n - \alpha| < \epsilon \bigr]\]
が成り立つことをいう。

ディリクレの関数
\begin{equation*}
f(x) = \lim_{n \to \infty} \lim_{k \to \infty} \cos^{2k}(n! \pi x)
\end{equation*}
\[
A = \left(
\begin{matrix}
a_{1,1} & a_{1,2} & \dots  & a_{1,n} \\
a_{2,1} & a_{2,2} & \dots  & a_{2,n} \\
\vdots  & \vdots  & \ddots & \vdots  \\
a_{m,1} & a_{m,2} & \dots  & a_{m,n}
\end{matrix}
\right)
\]

\end{document}
